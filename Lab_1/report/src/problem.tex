\section{Условие}
        Родительский процесс создает два дочерних процесса. Первой строкой пользователь в консоль
родительского процесса вводит имя файла, которое будет использовано для открытия File с таким
именем на запись для child1. Аналогично для второй строки и процесса child2. Родительский 
и
дочерний процесс должны быть представлены разными программами. Родительский процесс принимает от пользователя строки произвольной длины и пересылает их в
pipe1 или в pipe2 в зависимости от правила фильтрации. Процесс child1 и child2 производят работу
над строками. Процессы пишут результаты своей работы в стандартный вывод. 

\subsection*{Цель работы}
        Изучение механизмов создания процессов, организации межпроцессного взаимодействия через pipes и обработки данных в многопроцессной архитектуре.

\subsection*{Задание}
    Правило фильтрации: нечетные строки отправляются в pipe1, четные в pipe2.
    Дочерние процессы инвертируют строки.

\subsection*{Вариант}21