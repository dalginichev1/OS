\section{Результаты тестирования}

\subsection*{Методика тестирования}
Программа была протестирована на массивах трех размеров: 1 миллион, 5 миллионов и 10 миллионов элементов. Для каждого размера массива измерялось время выполнения при различном количестве потоков (от 1 до 64).

\subsection*{Результаты для массива 1 млн элементов}

\begin{itemize}
\item Максимальное ускорение: $\approx$9 раз
\item Оптимальное количество потоков: 4-8
\item Эффективность при 4 потоках: $\approx$50\%
\item Насыщение производительности: после 16 потоков
\end{itemize}

\begin{figure}[h]
\centering
\includegraphics[width=0.8\textwidth]{~/OS/Lab_2/best_graph/speedup_1млн.png}
\caption{График ускорения (массив 1 млн элементов)}
\label{fig:speedup_1m}
\end{figure}

\begin{figure}[h]
\centering
\includegraphics[width=0.8\textwidth]{~/OS/Lab_2/best_graph/efficiency_1млн.png}
\caption{График эффективности (массив 1 млн элементов)}
\label{fig:efficiency_1m}
\end{figure}

\subsection*{Результаты для массива 5 млн элементов}

\begin{itemize}
\item Максимальное ускорение: $\approx$10.3 раз
\item Оптимальное количество потоков: 8-16
\item Эффективность при 8 потоках: $\approx$50\%
\item Насыщение производительности: после 32 потоков
\end{itemize}

\begin{figure}[h]
\centering
\includegraphics[width=0.8\textwidth]{~/OS/Lab_2/best_graph/speedup_5млн.png}
\caption{График ускорения (массив 5 млн элементов)}
\label{fig:speedup_5m}
\end{figure}

\begin{figure}[h]
\centering
\includegraphics[width=0.8\textwidth]{~/OS/Lab_2/best_graph/efficiency_5млн.png}
\caption{График эффективности (массив 5 млн элементов)}
\label{fig:efficiency_5m}
\end{figure}

\subsection*{Результаты для массива 10 млн элементов}

\begin{itemize}
\item Максимальное ускорение: $\approx$14 раз
\item Оптимальное количество потоков: 16-32
\item Эффективность при 16 потоках: $\approx$50\%
\end{itemize}

\begin{figure}[h]
\centering
\includegraphics[width=0.8\textwidth]{~/OS/Lab_2/best_graph/speedup_10млн.png}
\caption{График ускорения (массив 10 млн элементов)}
\label{fig:speedup_10m}
\end{figure}

\begin{figure}[h]
\centering
\includegraphics[width=0.8\textwidth]{~/OS/Lab_2/best_graph/efficiency_10млн.png}
\caption{График эффективности (массив 10 млн элементов)}
\label{fig:efficiency_10m}
\end{figure}

\subsection*{Сравнительный анализ}

\begin{itemize}
\item \textbf{Зависимость ускорения от размера массива}:
    \begin{itemize}
    \item 1 млн элементов: ускорение до 9 раз
    \item 5 млн элементов: ускорение до 10.3 раз
    \item 10 млн элементов: ускорение до 14 раз
    \end{itemize}

\item \textbf{Оптимальное количество потоков}:
    \begin{itemize}
    \item 1 млн: 4-8 потоков
    \item 5 млн: 8-16 потоков
    \item 10 млн: 16-32 потока
    \end{itemize}

\item \textbf{Эффективность использования ресурсов}:
    \begin{itemize}
    \item Меньшие массивы достигают пиковой эффективности при меньшем количестве потоков
    \item Большие массивы позволяют эффективно использовать больше потоков
    \end{itemize}
\end{itemize}

\subsection*{Выводы}

\begin{itemize}
\item Программа демонстрирует хорошую масштабируемость, особенно для больших массивов данных
\item Наблюдается четкая зависимость: чем больше размер массива, тем больше потоков можно эффективно использовать
\item Максимальное ускорение достигает 14 раз для массива из 10 миллионов элементов
\item Для практического применения рекомендуется выбирать количество потоков в зависимости от размера данных
\end{itemize}