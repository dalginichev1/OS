\section{Метод решения}
Для реализации параллельной сортировки используется модифицированный алгоритм QuickSort:
\begin{itemize}
    \item Массив разбивается на части по количеству потоков
    \item Каждая часть сортируется в отдельном потоке
    \item После завершения всех потоков выполняется слияние отсортированных частей
\end{itemize}

Ключевые компоненты программы:
\begin{itemize}
    \item \texttt{QuickSort.hpp} - заголовочный файл с объявлением класса ParallelQuickSort
    \item \texttt{main.cpp} - точка входа, взаимодействие с пользователем
    \item \texttt{quicksort.cpp} - реализация параллельного алгоритма сортировки
    \item \texttt{Thread.hpp/cpp} - обертка для работы с потоками POSIX
\end{itemize}

\section{Описание программы}

\subsection*{Архитектура программы}
Программа использует объектно-ориентированный подход. Основной класс \texttt{ParallelQuickSort} инкапсулирует всю логику параллельной сортировки:

\begin{itemize}
    \item Разделение массива на части для параллельной обработки
    \item Создание и управление потоками
    \item Реализация алгоритма QuickSort
    \item Слияние отсортированных частей
\end{itemize}

\subsection*{Основные функции}

\textbf{Класс ParallelQuickSort:}
\begin{itemize}
    \item \texttt{partition()} - разделение массива относительно опорного элемента
    \item \texttt{quicksort()} - рекурсивная реализация алгоритма QuickSort
    \item \texttt{quicksort\_thread()} - функция для выполнения в потоке
    \item \texttt{sort()} - основной метод, организующий параллельную сортировку
\end{itemize}

\textbf{Класс Thread:}
\begin{itemize}
    \item \texttt{create()} - создание потока
    \item \texttt{join()} - ожидание завершения потока
\end{itemize}