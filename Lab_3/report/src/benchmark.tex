\section{Результаты тестирования}

\subsection{Пример работы программы}

\begin{verbatim}
Введите имя файла для child1: file1
Введите имя файла для child2: file2
child1: запущен с shared memory /child1_shm и выходным файлом file1
child2: запущен с shared memory /child2_shm и выходным файлом file2
hello
big
create
table
Данные отправлены в дочерние процессы
child1: получено 'hello\ncreate\n'
child2: получено 'big\ntable\n'
child1: обработал и записал данные в file1
child2: обработал и записал данные в file2
Данные обработаны дочерними процессами
Дочерние процессы завершили работу
\end{verbatim}

\subsection{Содержимое выходных файлов}

\textbf{file1:}
\begin{verbatim}
olleh
etaerc
\end{verbatim}

\textbf{file2:}
\begin{verbatim}
gib
elbat
\end{verbatim}

\subsection{Анализ системных вызовов}

В ходе работы программы были использованы следующие ключевые системные вызовы:

\begin{itemize}
    \item \texttt{shm\_open} - создание/открытие разделяемой памяти
    \item \texttt{mmap} - отображение памяти в адресное пространство процесса
    \item \texttt{sem\_init} - инициализация семафоров
    \item \texttt{sem\_wait/sem\_post} - операции синхронизации
    \item \texttt{fork} - создание дочерних процессов
    \item \texttt{execve} - запуск исполняемых файлов
    \item \texttt{waitpid} - ожидание завершения дочерних процессов
\end{itemize}