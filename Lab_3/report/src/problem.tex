\section{Условие}
Составить и отладить программу, осуществляющую работу с процессами и взаимодействие между ними в одной из двух операционных систем. В результате работы программа (основной процесс) должен создать для решение задачи один или несколько дочерних процессов. Взаимодействие между процессами осуществляется через системные сигналы/события и/или через отображаемые файлы (memory-mapped files).
Необходимо обрабатывать системные ошибки, которые могут возникнуть в результате работы.


\subsection*{Цель работы}
        Изучение механизмов межпроцессного взаимодействия через разделяемую память (shared memory) и синхронизации процессов с использованием семафоров.

\subsection*{Задание}
    Реализовать программу, состоящую из родительского и двух дочерних процессов, взаимодействующих через разделяемую память. Родительский процесс распределяет вводимые строки между дочерними процессами по принципу четности: нечетные строки передаются первому процессу, четные - второму. Каждый дочерний процесс инвертирует полученные строки и записывает результат в указанный файл..

\subsection*{Вариант}21