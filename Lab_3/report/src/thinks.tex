\section{Выводы}

В процессе выполнения этой лабораторной работы я получил ценный практический опыт в нескольких ключевых областях:

\textbf{Межпроцессное взаимодействие} - я на практике освоил работу с разделяемой памятью (shared memory), что оказалось гораздо интереснее теоретического изучения. Особенно впечатлила скорость обмена данными между процессами по сравнению с другими механизмами IPC.

\textbf{Синхронизация процессов} - работа с семафорами показала мне, насколько важна правильная синхронизация в многопроцессных системах. Я столкнулся с проблемой взаимной блокировки (deadlock) и научился ее диагностировать и исправлять, что было ценным опытом отладки.

\subsection{Приобретенные навыки}

\begin{itemize}
    \item \textbf{Практическое применение IPC} - теперь я уверенно могу использовать разделяемую память в своих проектах
    \item \textbf{Отладка многопроцессных приложений} - научился использовать strace, добавлять диагностику и анализировать сложные сценарии взаимодействия процессов
    \item \textbf{Проектирование архитектуры} - понял, как правильно разделять ответственность между процессами и организовывать их взаимодействие
\end{itemize}

\subsection{Заключение}

Эта лабораторная работа стала для меня настоящим прорывом в понимании многозадачности и межпроцессного взаимодействия. Из абстрактных концепций эти темы превратились в конкретные инструменты, которые я теперь могу применять в реальных проектах.

Особую ценность имел процесс преодоления трудностей - каждая исправленная ошибка давала новое понимание работы операционной системы. Я не просто написал программу, а глубоко разобрался в том, как она взаимодействует с ядром ОС.

Полученные знания и навыки обязательно пригодятся мне в будущем, особенно при работе с высоконагруженными приложениями и распределенными системами. Эта работа показала мне, что системное программирование - это не только сложно, но и невероятно интересно!