\section{Архитектура проекта}
Архитектура проекта реализует клиент-серверную модель с использованием разделяемой памяти (shared memory) для межпроцессного взаимодействия. Система состоит из трёх ключевых компонентов, определённых в файлах проекта:
Сервер (Server.hpp, Server.cpp) — центральный координатор, который управляет всеми игровыми сессиями, обрабатывает команды клиентов и поддерживает целостность игрового состояния. Сервер работает в бесконечном цикле, ожидая сообщений в общей очереди.
Клиенты (Client.hpp, Client.cpp) — независимые процессы, предоставляющие консольный интерфейс игрокам. Каждый клиент подключается к существующей разделяемой памяти, регистрируется с уникальным логином и взаимодействует с системой через меню.
Разделяемая память (SharedTypes.hpp, SharedMemory.hpp, SharedMemory.cpp) — общая область памяти, содержащая структуру SharedMemoryRoot с очередью сообщений, слотами клиентов и массивами игр. Этот компонент обеспечивает высокоскоростное взаимодействие между процессами.
Все компоненты взаимодействуют через единую структуру данных, определённую в SharedTypes.hpp, которая включает типы сообщений (MsgType), состояния ячеек (CellState), кораблей (Ship) и игр (GameData). Синхронизация осуществляется с помощью POSIX мьютексов и условных переменных, также определённых в структуре разделяемой памяти.

\section{Метод решения}
Метод решения основан на использовании POSIX Shared Memory для реализации межпроцессного взаимодействия между сервером и клиентами. В файле SharedTypes.hpp определена центральная структура SharedMemoryRoot, содержащая:
Очередь сообщений (Message queue[QUEUE_SIZE]) — циклический буфер для асинхронной передачи команд от клиентов серверу и обратно. Типы сообщений определены в перечислении MsgType и включают все возможные действия: регистрация (MSG_REGISTER), создание игры (MSG_CREATE), выстрел (MSG_SHOT) и др.
Слоты клиентов (ClientSlot clients[MAX_CLIENTS]) — информация о каждом подключённом игроке, включая логин, условную переменную для ожидания ответов и текущий статус.
Игровые сессии (GameData games[16]) — полное состояние каждой игры, включая поля игроков (board1, board2), массивы кораблей (ships1, ships2), статистику и временные метки.
Серверный процесс (реализованный в Server.cpp) работает по следующему алгоритму:
Инициализирует разделяемую память и создаёт необходимые объекты синхронизации.
Входит в основной цикл, ожидая сообщений через условную переменную server_cond.
При получении сообщения определяет его тип и вызывает соответствующий обработчик (handle_message()).
Обновляет состояние игр, отправляет ответы клиентам и синхронизирует изменения.
Клиентский процесс (Client.cpp) реализует:
Подключение к существующей разделяемой памяти.
Регистрацию с уникальным логином через сообщение MSG_REGISTER.
Взаимодействие с пользователем через контекстные меню (главное меню, меню расстановки, игровое меню).
Отправку команд в очередь и ожидание ответов через условные переменные.
Игровая логика инкапсулирована в классе Game (Game.hpp, Game.cpp), который отвечает за:
Валидацию расстановки кораблей (проверка границ, пересечений, расстояний).
Обработку выстрелов с определением попаданий, промахов и уничтожения кораблей.
Управление очерёдностью ходов (правило дополнительного хода при попадании).
Определение условия победы (уничтожение всех кораблей противника).
Синхронизация обеспечивается через мьютекс mutex в структуре SharedMemoryRoot. Каждая операция чтения или записи в разделяемую память предваряется захватом этого мьютекса, что предотвращает состояния гонки (race conditions). Условные переменные (server_cond и cond в каждом ClientSlot) используются для эффективного ожидания событий без активного опроса.
Обработка ошибок реализована на всех уровнях: проверка корректности входных данных (координаты в пределах 0-9, правильный формат команд), валидация игровых действий (очередность хода, завершённость расстановки), обработка исключительных ситуаций (отключение клиента, переполнение очереди).

\section{Описание программы}
Сервер запускается первым и выполняет следующие функции:
Создаёт и инициализирует разделяемую память (если не существует).
Устанавливает структуры данных: очередь сообщений, слоты клиентов, игровые сессии.
Инициализирует мьютексы и условные переменные с атрибутами PTHREAD_PROCESS_SHARED.
Ожидает команды от клиентов в бесконечном цикле, обрабатывая их в порядке поступления.
Управляет жизненным циклом игр: создание, наполнение игроками, проведение игры, завершение и очистка.
Обеспечивает целостность данных и соблюдение правил игры.

\subsection{Клиент (client)}
Клиентское приложение предоставляет пользователю консольный интерфейс с контекстными меню:
Главное меню (отображается когда игрок не в игре):
Список игроков и игр (команда list → MSG_LIST)
Создание публичной игры (ввод имени → MSG_CREATE)
Присоединение к игре по имени или ID (MSG_JOIN)
Приглашение другого игрока по логину (MSG_INVITE)
Проверка приглашений
Выход из системы (MSG_QUIT)
Меню расстановки кораблей (после входа в игру):
Ручное размещение: place размер,x,y,ориентация → MSG_PLACE_SHIP
Автоматическая расстановка: auto (алгоритм в Client.cpp)
Просмотр своего поля: board → MSG_GET_BOARD
Завершение расстановки: ready → MSG_SETUP_COMPLETE
Приглашение в текущую игру: invite логин → MSG_INVITE_TO_GAME
Выход из игры: menu → MSG_LEAVE_GAME

\subsection{Игровое меню (после начала боя)}
Сделать выстрел: ввод координат → MSG_SHOT
Просмотр своего поля → MSG_GET_BOARD
Просмотр поля противника → MSG_GET_OPPONENT_BOARD
Статус игры → MSG_GAME_STATUS
Сдаться → MSG_SURRENDER
Выйти в меню → MSG_LEAVE_GAME

\subsection{Игровой процесс}
Игра следует классическим правилам "Морского боя":
Поле 10×10 клеток.
Флот из 10 кораблей: 1×4, 2×3, 3×2, 4×1 клетки.
Расстановка с расстоянием минимум 1 клетка между кораблями.
Поочерёдные ходы с правом дополнительного хода при попадании.
Отображение полей: своё поле показывает корабли, поле противника скрывает неподбитые корабли.
Победа при уничтожении всех кораблей противника.

\section{Особенности реализации}
Автоматическая расстановка использует алгоритм случайного размещения с проверкой корректности позиции.
Система приглашений поддерживает два сценария: создание новой приватной игры и приглашение в существующую игру.
Визуализация полей с помощью символов ASCII: . — пусто, S — корабль (на своём поле), X — попадание, O — промах, # — потопленный корабль.
Обработка отключений: сервер обнаруживает неактивных клиентов и корректно завершает игры.
Масштабирование: ограничение на 32 одновременных клиента и 16 активных игр (константы в SharedTypes.hpp).