\section{Выводы}

В процессе выполнения лабораторной работы я получил практический опыт работы с динамическими библиотеками и двумя типами их использования.

\textbf{Статическое связывание} - библиотеки линкуются на этапе компиляции. Преимущества: высокая производительность, простота развертывания. Недостатки: большой размер исполняемого файла, невозможность обновления без перекомпиляции.

\textbf{Динамическая загрузка} - библиотеки загружаются во время выполнения. Преимущества: гибкость, возможность обновления, разделение кода. Недостатки: дополнительная сложность, overhead на загрузку.

\subsection{Приобретенные навыки}

\begin{itemize}
    \item \textbf{Работа с dlopen/dlsym} - освоил механизм динамической загрузки в Linux
    \item \textbf{Архитектура плагинов} - научился проектировать системы с подключаемыми модулями
    \item \textbf{Создание shared libraries} - изучил процесс сборки .so файлов
\end{itemize}

\subsection{Заключение}

Динамическая загрузка библиотек открывает большие возможности для создания гибких и расширяемых приложений. Несмотря на дополнительную сложность, этот подход незаменим в случаях, когда требуется:
\begin{itemize}
    \item Поддержка плагинов и расширений
    \item Обновление компонентов без перекомпиляции
\end{itemize}

Полученные знания позволяют создавать более архитектурно сложные и поддерживаемые приложения.