\section{Выводы}

Анализ системных вызовов с помощью утилиты \texttt{strace} позволил получить глубокое практическое понимание работы операционной системы Linux. 

\textbf{Ключевые выводы:}

1. \textbf{Межпроцессное взаимодействие} — изучены различные механизмы IPC: от простых pipes до разделяемой памяти с синхронизацией через futex. Практически подтверждены различия в производительности и сложности реализации.

2. \textbf{Архитектура процессов/потоков} — на уровне системных вызовов увидел разницу между процессами (\texttt{clone()}) и потоками (\texttt{clone()} с \texttt{CLONE\_THREAD}). Понял преимущества потоков для параллельных вычислений.

3. \textbf{Динамические библиотеки} — проанализирован процесс загрузки и выполнения ELF-файлов, разница между статической и динамической линковкой, возможности горячей замены библиотек.

4. \textbf{Инструментарий} — освоил использование \texttt{strace} как мощного средства отладки, позволяющего понять реальное поведение программ на уровне ядра ОС.

Полученные знания формируют прочную основу для разработки эффективных системных приложений и понимания внутренней работы операционных систем.